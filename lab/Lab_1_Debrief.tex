\documentclass{beamer}
\usepackage{url}
\usepackage{verbatim}
\usepackage{listings}

\lstset{basicstyle=\ttfamily \scriptsize,
  basicstyle=\ttfamily,
   columns=fullflexible,
   breaklines=true,
   numbers=left,
   numberstyle=\scriptsize,
   stepnumber=1,
   mathescape=false,
   tabsize=2,
   showstringspaces=false
}

\usetheme[compress]{Berlin}

\begin{document}

\title{Lab 1 Debrief}
\author[Rollen]{Rollen D'Souza \\ \url{rs2dsouz@uwaterloo.ca}}
\date{06 October 2016}
\institute{University of Waterloo}

\begin{frame}
	\titlepage
\end{frame}

\begin{frame}
	\frametitle{Outline}
	\tableofcontents
\end{frame}

\section{General Feedback}
\subsection{Source Control}
\begin{frame}
	\frametitle{Using Subversion}
	\begin{itemize}
		\item Setup your SVN \alert{before} creating a project! \pause
		\item Commit code that compiles. \pause
		\item Commits should add/remove features or fix bugs, in general. \pause
		\item Please, only one copy of your assignment.
	\end{itemize}
\end{frame}
\begin{frame}[fragile]
	\frametitle{Why only one copy}
	\begin{example}
		\texttt{svn log . -v}
		\begin{verbatim}
r1138 | | 2016-09-29 11:29:26 -0400 

r1137 | | 2016-09-29 11:28:58 -0400

r1136 | | 2016-09-29 11:28:33 -0400
		\end{verbatim}
	\end{example}
	\pause
	\begin{example}
		\texttt{svn update -r 1136}\\
		\texttt{svn update -r \{2016-10-06T11:30\}}
	\end{example}
\end{frame}

\subsection{C Programming}
\begin{frame}[fragile]
	\frametitle{C Arrays and Strings}
	What is wrong with the following piece of code?
	\begin{example}
		\begin{lstlisting}[language=C]
			char buffer[500];
			readIntoBuffer(buffer);
			printf(buffer);
		\end{lstlisting} 
	\end{example}
	\pause
	What if \texttt{readIntoBuffer} didn't write to \texttt{buffer}?
\end{frame}
\begin{frame}[fragile]
	\frametitle{C Arrays and Strings}
	Always initialize your arrays!
	\begin{example}
		\begin{lstlisting}[language=C]
			char buffer[500] = { 0 };
			readIntoBuffer(buffer);
			printf(buffer);
		\end{lstlisting}
	\end{example}
\end{frame}

\begin{frame}[fragile]
	\frametitle{Switches}
	\begin{example}
	\begin{lstlisting}[language=C]
		switch(character)
		{
		case 'A':
		case 'a':
			// code here.
			break;

		case 'B':
		case 'b':
			// more code here.
			break;
		}
	\end{lstlisting}
	\end{example}
\end{frame}

\section{Lab Specifics}
\subsection{Lab Solution}
\begin{frame}
	\frametitle{Lab Solution}
	See Code in Repository.
\end{frame}

\subsection{Advice for the Future}
\begin{frame}
	\frametitle{Solution Take-Aways}
	\begin{itemize}
		\item Tempted to hard-code a number? Use a \texttt{\#define} or \texttt{const int} instead. \pause
		\item{Use more than one file.
			\begin{itemize}
				\item \texttt{extern}
				\item \texttt{forward declarations}
			\end{itemize}
		} \pause
		\item Functions are for reusable code. \pause
		\item{Use the C Standard Library \\ See \url{http://en.cppreference.com/w/c}}
	\end{itemize}
\end{frame}

\begin{frame}
	\frametitle{Advice}
	\begin{itemize}
		\item Be Careful and Methodical. \pause
		\item{Ask for help early!
			\begin{itemize}
				\item Piazza
				\item Email
				\item Your Peers
			\end{itemize}
		} \pause
		\item If you are unsure how to make your code better, talk it out!
	\end{itemize}
\end{frame}

\end{document}
