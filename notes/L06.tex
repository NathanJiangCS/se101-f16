\documentclass[11pt]{article}
\usepackage[utf8]{inputenc}
\usepackage{textcomp}
\usepackage{listings}
\usepackage{tikz}
\usepackage{enumerate}
\usepackage{url}
%\usepackage{algorithm2e}
\usetikzlibrary{arrows,automata,shapes}
\tikzstyle{block} = [rectangle, draw, fill=blue!20, 
    text width=5em, text centered, rounded corners, minimum height=2em]
\tikzstyle{bt} = [rectangle, draw, fill=blue!20, 
    text width=1em, text centered, rounded corners, minimum height=2em]

\lstset{ %
  basicstyle=\ttfamily,commentstyle=\scriptsize\itshape,showstringspaces=false,breaklines=true,numbers=none}

\newtheorem{defn}{Definition}
\newtheorem{crit}{Criterion}

\newcommand{\handout}[5]{
  \noindent
  \begin{center}
  \framebox{
    \vbox{
      \hbox to 5.78in { {\bf Intro to Methods of Software Engineering } \hfill #2 }
      \vspace{4mm}
      \hbox to 5.78in { {\Large \hfill #5  \hfill} }
      \vspace{2mm}
      \hbox to 5.78in { {\em #3 \hfill #4} }
    }
  }
  \end{center}
  \vspace*{4mm}
}

\newcommand{\lecture}[4]{\handout{#1}{#2}{#3}{#4}{Lecture #1}}
\topmargin 0pt
\advance \topmargin by -\headheight
\advance \topmargin by -\headsep
\textheight 8.9in
\oddsidemargin 0pt
\evensidemargin \oddsidemargin
\marginparwidth 0.5in
\textwidth 6.5in

\parindent 0in
\parskip 1.5ex
%\renewcommand{\baselinestretch}{1.25}

\newcommand{\squishlist}{
 \begin{list}{$\bullet$}
  { \setlength{\itemsep}{0pt}
     \setlength{\parsep}{3pt}
     \setlength{\topsep}{3pt}
     \setlength{\partopsep}{0pt}
     \setlength{\leftmargin}{1.5em}
     \setlength{\labelwidth}{1em}
     \setlength{\labelsep}{0.5em} } }
\newcommand{\squishlisttwo}{
 \begin{list}{$\bullet$}
  { \setlength{\itemsep}{0pt}
     \setlength{\parsep}{0pt}
    \setlength{\topsep}{0pt}
    \setlength{\partopsep}{0pt}
    \setlength{\leftmargin}{2em}
    \setlength{\labelwidth}{1.5em}
    \setlength{\labelsep}{0.5em} } }
\newcommand{\squishend}{
  \end{list}  }

\begin{document}

\lecture{6 --- October 25, 2016}{Fall 2016}{Patrick Lam}{version 1}

\section*{Real-World Software Engineering Practices}

Students are sometimes tempted to collaborate by emailing their source
code around. This is terrible! Imagine doing that when you have 100
source files and 20,000 lines of code. Don't do it.

Today, then, I'll talk about ways that companies work together in teams
to ship software. Not all companies do all of the things here, but they are
best practices.

This particular formulation is the ``Joel Test'', which Joel Spolsky
proposed in 2000:
\url{http://www.joelonsoftware.com/articles/fog0000000043.html}.
We'll apply a modified Joel Test to your Capstone Design Project.

\paragraph{The Joel Test: 12 Steps to Better Code.}

\begin{enumerate}
\item Do you use source control?\\
  I've seen that some of you use github. This is one popular source control system.
  Source control enables you to collaborate with others productively, and also to
  go backwards in time to an older version of your code.
\item Can you make a build in one step?\\
  Should be: check out source code, build, deploy. As a software engineer, automation
  is your best friend. Automating builds eliminates errors and allows fast iteration.
\item Do you make daily builds?\\
  This avoids blocking your teammates and enforces the norm that your system is generally in a working state (not ``I'll fix it later.'')
\item Do you have a bug database?\\
  You'll forget about bugs that are not in the bug database.
\item Do you fix bugs before writing new code?\\
  Better to minimize interval between bug creation and bug fixing when possible; older bugs are
  harder to fix, because you forgot the context.
\item Do you have an up-to-date schedule?\\
  Joel Spolsky proposes ``Evidence-Based Scheduling'', which amounts to recording estimates and tracking how close you are to your estimates. Estimates can be notoriously unreliable.
\item Do you have a spec?\\
  Make sure you are implementing the right thing. It's easier to change the specification than the code. (Prototypes/mockups are useful, though.)
\item Do programmers have quiet working conditions?\\
  The concept of ``flow'' is super important for writing code. Typically it'll take 15 minutes to get started. If you're interrupted, then you need another 15 minutes to get started again, which is a huge drain on productivity. People often wear headphones to code, but is that really the best thing?
\item Do you use the best tools money can buy?\\
  In the corporate context, tools are cheap, while developers are expensive. (Note that your salary is only about one-third the cost of employing you.)
\item Do you have testers?\\
  Automated testing has evolved a lot since this article was written; best practice today is to never use tests driven by humans according to a script (although some of you will probably still have such jobs on your first co-op term).

  On the other hand, exploratory testing is still a best practice today. Testers can be better than developers at exploratory testing, since they didn't write the software.
\item Do new candidates write code during their interview?\\
  You can see why this is useful from the company point of view. Let me address the other side. What if you're in SE and you can't yet write code? Consider this article: Mark Guzdial. ``Anyone can learn programming: Teaching $>$ Genetics,'' CACM Blog. \url{http://cacm.acm.org/blogs/blog-cacm/179347-anyone-can-learn-programming-teaching-genetics/fulltext}.
\item Do you do hallway usability testing?\\
  As with testers: get the software in front of someone else as soon as possible. One always has
  trouble seeing problems in one's own writing/designs/etc.
\end{enumerate}

Again, the Joel Test is most effective for evaluating a team's effectiveness.
A score of 12 is perfect; Joel writes that a score of 11 is tolerable; and yet
companies do operate at scores of 2 or 3.

A high Joel Test score is necessary but not sufficient for writing
useful software, though. No matter how good your process, if you're not solving
someone's problem (product/market fit), your software won't get used much.

\section*{Software Engineering core courses}

In some ways, the 3-course sequence SE 465 (3A)/SE 464 (3B)/SE 463 (4A)
is at the core of the ``software engineering'' content in your curriculum.

\paragraph{SE 465 (Testing).} There is no magic bullet for producing test cases,
but there are tools that help, as well as some concepts that you can understand.
\paragraph{SE 464 (Architecture).} Helps you think about designs and how to put together
the pieces of your applications.
\paragraph{SE 463 (Requirements).} Got to solve the right problem.



\end{document}
